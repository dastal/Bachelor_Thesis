\chapter{Introduction} \label{introduction}

\section{Motivation} \label{motivation}
Conventional DNS queries entail the problem that not only the user and the resolver, but also nearly everyone can see the content of those queries. Therefore, DNS over HTTPS (DoH) was introduced in October 2018 with the goal to improve the internet security and user privacy by sending encrypted DNS queries using the HTTPS protocol. In September 2019, Firefox announced to adopt DoH into their browser \cite{TweetFirefox} (see image \ref{fig:tweet_ghacks}) and since then the usage of this protocol experienced a steep increase \cite{GarciaEtAl_LargeScaleMEasurement}, although other security approaches still form the vast majority.

\begin{figure} [h]
\frame{\includegraphics[scale=0.75]{tweet_ghacks}}
\centering
\caption{Tweet \cite{TweetFirefox} in which the Implementation of DoH was introduced}
\label{fig:tweet_ghacks}
\end{figure}

A huge benefit of DoH is that it protects the content of the traffic from the insight of third parties. But exactly because of this point experts have expressed concerns \cite{Blog_PrevalenceOfDoH}, since it is also not possible for DNS monitoring systems to have a direct insight into the traffic. This complicates the inspection, the detection, and the blocking of DoH traffic and makes it a difficult task. Therefore, DoH brings along not only desired properties, but also many undesired properties, like bypassing of DNS monitoring Systems or exploitation of upstream DNS traffic \cite{NSA_AdoptingEncryptedDNS}. One example for this type of misusage is the Godlua Backdoor Malware \cite{GodluaBackdoor} which uses DoH to obtain the address of the C2. A more recent example for the misuse of DoH is the Iranian hacker group Oilrig (APT34) \cite{Oilrig}, which used DoH in combination with the tool DNSExfiltrator \cite{DNSExfiltrator} to receive data without being noticed form a hacked network.

The encryption of DoH traffic makes direct deep inspection for malware detection a nearly impossible task. Nevertheless, there exists another way to get an insight in the data traffic of DoH: leveraging the unique traffic shape. There exist approaches where the traffic is successfully analyzed using machine learning models. The SecGrid \cite{SecGrid} system developed at the University of Zurich by the CSG Group provides a platform for granular feature extraction. The aim of this thesis is to implement a prototypical malicious DoH traffic detection component into SecGrid.


\section{Description of Work} \label{description}
This thesis is separated into three different stages: the first stage is the introduction into the problem, in the second stage a literature research is conducted and the findings of it are presented, and into the final stage a working prototype is created, evaluated, and finally the whole work is documented.

The first stage initially comprises the fundamental understanding of DoH. Therefore a deep insight into DNS, its encryption, and the detection of encrypted DNS queries is to be achieved to understand the underlying problem of my work to fulfil the objectives of my thesis. Further, the insight into the baseline system with its background will be the predominant part of this stage. The platform will be analyzed to get an overview of the existing system, the feature extraction and the current machine learning based clients that are already implemented into the system. The final step of this stage will be the decomposition of the thesis into tasks, such that a timeline can be established where the systematic implementation of the work can be planned.

In the second stage the current state of the literature is surveyed. The first step is a research about the background of DoH to get a detailed insight into the objective. This includes a survey about the specific characteristics of malware that misuse the security properties of DoH for cyberattacks. The main part of this stage will be a survey about the current approaches in the detection of malicious DoH traffic, with special focus on machine learning based approaches. In the subsequent part of this stage the findings achieved in the literature research are presented, followed by the main design decisions for the thesis. Finally, the findings are adopted into the current system architecture. Therefore the current platform architecture is redesigned and the findings are implemented to show the technical feasibility and the usefulness of the work. 

The third and last stage contains the implementation of a working prototype, where the findings and the design decisions of the second stage are implemented into the platform. The prototype contains a working DoH traffic detection component. The final part of this stage is the evaluation and the documentation of the work.


\section{Thesis Outline} \label{outline}
This report is structured as follows. Based on formulation of the thesis goal in this Chapter \ref{introduction}, Chapter \ref{backgorund} delivers the background knowledge on which this thesis is built. The focus there is laid to the DNS-over-HTTPS protocol, on its basics, strengths, vulnerabilities, and malware which is already abusing it. Furthermore SecGrid is presented in more detail, including the file-format of the Packet-Capture. Chapter \ref{related_work} is dedicated to the findings of the literature research. Thus, the focus there is laid on current solutions for the detection of malicious DNS-over-HTTPS traffic, with special interest on Machine Learning based approaches. In Chapter \ref{design}, the design of the architecture of the prototype for malicious DNS-over-HTTPS traffic detection, which is implemented into SecGrid is presented. Chapter \ref{implementation} treats the implementation of the prototype into SecGrid step by step. Foremost, the Feature Extraction is described, followed by the extraction of the data-set which was used for the training data of the Machine Learning model and ultimately the Machine Learning model implementation is described. In Chapter \ref{evaluation} the evaluation of the prototype is conducted, whereas the Extraction and the ML Model are focused. Finally, Chapter \ref{summary} concludes the work and the findings of this thesis, the conclusions and limitations are presented and future work is stated.
